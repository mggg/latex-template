\documentclass{mgggarticle}
\usepackage[utf8]{inputenc}
\usepackage[english]{babel}
\usepackage{hyperref}

% This document is an example for how to use this LaTeX template.

\title{Comparison of Districting Plans\\ for the Virginia House of Delegates}
\author{Metric Geometry and Gerrymandering Group}
\date{November 2018}

\begin{document}

\begin{titlepage}

\maketitle

\begin{abstract}
At the time of writing, Virginia is in the process of replacing its House of Delegates districting
plan after eleven of the districts were ruled unconstitutional by a District Court in June 2018.
This report presents a large ensemble of alternative valid districting plans, which we propose
to use as a baseline for comparison in the evaluation of newly proposed plans. Our method
highlights and quantifies the dilutive effects of packing Black Voting Age Population.
This is a novel application to racial gerrymandering of industry-standard techniques from
statistics and computational science.
\end{abstract}

\tableofcontents

\end{titlepage}

\decoratedsection{Summary Report}

\subsection{Introduction}

Using a mathematical sampling technique called a Markov chain, we have analyzed numerous
alternatives for districting the portion of the Virginia House of Delegates map that is affected by
the federal court ruling of June 26, 2018 \footnote{In this decade, the drawing of political boundaries in Virginia was particularly
contentious, with a significant number of lawsuits at all levels. A detailed summary of this process can be found at
\url{https://ballotpedia.org/Redistricting_in_Virginia_after_the_2010_census}.}. The results of our trials show the power and flexibility
of approaches from mathematics and computing for assessing proposed maps in a redistricting
process. A technical report detailing the findings follows below. This section serves as a nontechnical
overview of findings.

\subsection{Scope and goal for study}

In a June decision of the Eastern District Court of Virginia, 11 House districts were found to be
unconstitutional; by expanding to the districts neighboring those, we arrive at a minimum of 33
House districts out of 100 that must be reexamined \footnote{Bethune–Hill v. Virginia Board of Elections}.
The majority of the three-judge panel found that racial considerations had predominated over
traditional districting principles in the plan enacted in 2011, in particular, that Black residents
were concentrated in a way that diluted their voting strength.

We have analyzed a collection of different House plans: the enacted plan approved by the Legislature
in 2011 as House Bill 5005, the Democratic Caucus Plan released in August as HB 7001, the
“reform map” independently released by the Princeton Gerrymandering Project on Aug 28, 2018,
and a sequence of three plans circulated by Republican legislators in September and October. For
each of these, our analysis focuses on the distribution of Black Voting Age Population, or BVAP,
across the districts, though we note that this analysis can equally well be extended to focus on
partisan performance, preservation of city boundaries, or other quantifiable priorities.

It was confirmed in statements to the court that the 2011 Enacted plan was designed to have
$\geq55\%$ BVAP in 11 districts, which we may call the packed districts, without an effort to justify that
numerical cutoff on the level of the individual districts. Compliance with the Voting Rights Act in
no way requires this or any other numerical BVAP level. The conspicuous elevation of BVAP in the
11 packed districts must necessarily cause depressed BVAP in the 22 neighboring districts. We set
out to measure the pattern of that dilutive effect.

\paragraph{Example Paragraph.} Without a baseline for the BVAP distribution, it is impossible to assess whe\-ther the reduced BVAP
occurs in patterns that indicate a cost in additional opportunity districts, given the districting rules
and priorities. Our main goal is to observe and quantify the interplay between elevated BVAP in
part of the map and broader effects on the districting outcomes. The Markov chain method allows
us to construct a baseline for comparison, in order to quantify these tradeoffs.


\begin{figure}
    \centering
    \includegraphics[width=.7 \textwidth]{figure}
    \caption{%
        Comparison of the BVAP of the six proposed plans to an ensemble of neutral comparison maps that has been
        winnowed so each district has $\leq 60\%$ BVAP. The 33 districts of the affected region are arranged on the $x$
        axis, and the percentages of Black Voting Age Population are shown on the $y$ axis. The $37-55\%$ BVAP zone
        is marked in green, and the districts are separated into groups for ease of interpretation.
    }
    \label{fig:my_label}
\end{figure}

\subsection{Our main tool: an ensemble of valid alternative plans}

Using Markov chain techniques, we can apply long sequences of transformations to plans being
evaluated, and to neutral “seed” plans, to assemble large collections of districting plans that are
constructed only according to the stated rules and principles of a jurisdiction. For this study, our
ensemble is built to take into account traditional districting principles in play in Virginia. Our
Markov chain sampling process has population equality, contiguity, and compactness built into the
steps. In this way, we get a picture of how well the proposed plans comport with the principles
found in state and federal law and we gain a sense of how much these principles may be sacrificed
if other unstated goals are in play.

\subsection{Black Voting Age Population and the Crucial 37-55\% Range}

The recent lawsuit was brought under the federal Voting Rights Act of 1965, which continues to
be a fundamental check on redistricting practices. The range of BVAP values from 37\% and 55\%
is a crucial zone for VRA analysis nationally and in Virginia in particular. Though we emphasize
that BVAP alone is never enough to confirm VRA compliance, this wide range of BVAP values
often triggers the evaluation of racially polarized voting (RPV) patterns as one of several other
considerations addressing the totality of circumstances in a possible VRA violation. Nationally, 37\%
is an empirical bright line for congressional voting: 32 out of 34 current U.S. congressional districts
with at least 37\% BVAP had Representatives in the 115th Congress who belong to the Congressional
Black Caucus, and the ratio drops off precipitously below that level.5 Furthermore, Virginia-specific
data legitimates the significance of bracketing the 37-55\% range. The expert reports in Bethune–Hill
v. Virginia Board of Elections show no RPV evidence that any House districts in Virginia that would
require numbers over 55\% BVAP to comply with the Voting Rights Act, and the 2011 Enacted plan
was specifically faulted for aiming districts above the 55\% line. Indeed, the report of Maxwell
Palmer indicates that a BVAP of 45\% would suffice for all but one House district, and 48\% would
suffice for every House district in the state.6

In addition to Palmer’s RPV analysis, we note that
most of the 33-district affected region is covered by two congressional districts, VA-3 and VA-4,
which are comfortably electing Black representatives with BVAP of 45.1\% and 41.5\%, respectively.
It is important to remember that in some cases, less than 37\% BVAP can suffice for a district to
provide an opportunity to elect a candidate of choice of the Black community, especially in coalition
with other groups; in rare cases (particularly in other parts of the country), more than 55\% BVAP
may be legitimately required. Still, the figures below highlight the 37-55\% range to illustrate the
statistical effects of packing Black population: We can measure the cost of pushing districts above
that crucial zone by observing whether it causes other districts to be pushed below.

\subsection{Methods and Results}

We have constructed large collections of 33-district plans called ”ensembles,” defined on the region
of the state that is directly affected by the court ruling. To compare the BVAP levels among those
plans, we have indexed the 33 districts from the one with lowest BVAP (1st, appearing left-most in
the figure) to the one with highest BVAP (33rd). That means that in each plan, the twelve districts
with the highest levels of Black Voting Age Population are indexed 22-33, and are separated by a
dotted line in the figure. We have broken the districts into groups—top 12, next 4, next 9, next 4,
final 4—to make it easier to visualize where the effects of elevated BVAP occur.
Our algorithm begins with many different valid starting plans: the plans that have been proposed
for adoption and 100 neutral maps, labeled Seed1 through Seed100. We then perform chains
of random alterations, collecting a large sample from the resulting maps as our ensemble of
comparable plans.

Figure 1 shows one compelling set of results as a box-and-whisker plot. It depicts the ensemble
of 20,000 steps from a ”recombination” Markov chain, winnowed to show the plans that do not
exceed 60\% BVAP. (Many more figures with different combinations of hypotheses are described in
the technical report, lending robust support to similar conclusions.) The boxes show the 25th-75th
percentile of BVAP observed for a district in that position; the median is marked with a horizontal
line. The whiskers show the 1st-99th percentile of observations. When a colored dot falls far outside
of the whiskers, it means that the plan is an extreme outlier in its racial composition for that district.
The plot gives unmistakable evidence of where and how elevating BVAP in the top 12 districts
depresses BVAP in the remainder of the plans. Rather than reducing the BVAP in the areas where it
was already very low, we see that the dilutive effects impact districts that were at or nearing the
zone in which statistical analysis has indicated opportunities to elect more candidates of choice for
the Black community.

\subsection{Conclusions}

\begin{itemize}
    \item By starkly elevating BVAP in the six districts indexed 22-27, the 2011 Enacted plan causes at
least ten and up to 17 other districts to have depressed BVAP levels, far below what would be
expected from race-neutral redistricting. The 2011 Enacted plan has no districts at all in the
crucial range of 37\%-55\% BVAP, while neutral redistricting tells us to expect as many as ten.

    \item The Democratic Caucus plan and all new Republican proposals temper the packing in the top
districts but only push one additional district over 37\% BVAP.

    \item The Princeton plan—and hundreds of thousands of race-neutral plans found by Markov chain
techniques—push three additional districts over that level without sacrificing population balance,
contiguity, or compactness. In fact, our methods suggest that a substantial share of race-neutral
plans that comport with traditional districting principles would do so.
\end{itemize}

We emphasize that there are many local and community-based considerations in play when
approving districting plans, and the Markov chain approach only provides data relevant to some
of these. It does so by illustrating the range of properties typically observable in the enormous
landscape of valid plans. We view this approach as one tool among many in a complex process for
evaluating districting plans, and we hope it will incorporated into the analysis of proposed plans
to the House of Delegates. We are prepared to analyze new maps as they are proposed.

\begin{contributors}
Daryl R. DeFord, Moon Duchin, and Justin Solomon contributed to this report.
\end{contributors}

\end{document}
